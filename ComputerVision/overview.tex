% Created 2014-02-12 Mi 21:27
\documentclass[11pt]{article}
\usepackage[utf8]{inputenc}
\usepackage[T1]{fontenc}
\usepackage{fixltx2e}
\usepackage{graphicx}
\usepackage{longtable}
\usepackage{float}
\usepackage{wrapfig}
\usepackage[normalem]{ulem}
\usepackage{textcomp}
\usepackage{marvosym}
\usepackage{wasysym}
\usepackage{latexsym}
\usepackage{amssymb}
\usepackage{amstext}
\usepackage{hyperref}
\tolerance=1000
\usepackage[top=2cm, bottom=2cm, left=1cm, right=1cm]{geometry}
\usepackage{amsmath}
\author{Moe}
\date{\today}
\title{overview}
\hypersetup{
  pdfkeywords={},
  pdfsubject={},
  pdfcreator={Emacs 23.4.1 (Org mode 8.0.6)}}
\begin{document}

\maketitle
\tableofcontents


\section{Introduction:}
\label{sec-1}
\begin{itemize}
\item Sampling: discretisation of image domain
\item Quantisation: 
\begin{itemize}
\item discretation of co-domain   (value range)
\item Humans only see  40 different greyscales
\end{itemize}
\item Dimensions of image domain:
\begin{itemize}
\item 1 = Signal
\item 2 = Image
\item 3 = e.g CT scan (voxels)
\end{itemize}
\item Dimension of co-domain: eg. 1 for grayscale, 3 for RGB, frequency bands
\begin{itemize}
\item special case: co-domain from $\mathbb{R}^{n\times n} \rightarrow$  MRI scans
\end{itemize}
\item Imagesequences $\Rightarrow$ image domain +1 dimension
\end{itemize}

\section{Features and Descriptors}
\label{sec-2}
Features are "interessting" parts of the image

\subsection{Linear Diffusion}
\label{sec-2-1}
Physical phenomen where concentration differences are equalized is called diffusion

\begin{itemize}
\item Fick's Law:
$$ j = -D \cdot \nabla u $$
flux j, diffusion tensor D, u concentration
\item preservation of mass: $\partial u_t = -\text{div}(j)$
\item General diffusion equation:
$$ \partial u_t = \text{div}(D \nabla u) $$
with $\nabla u = (u_x, u_y)^T$
\item in linear case: $D = I \Rightarrow \partial u_t = \text{div}(\nabla u) = \Delta u$
\end{itemize}
\emph{Implementation Details:}
\begin{itemize}
\item \emph{original (greyscale) image = initial conditions}
\item \emph{reflecting boundary conditions $\rightarrow$ closed system, $\partial$$_{\text{n}}$ u = 0}
\end{itemize}

\subsubsection{Gaussian Convolution}
\label{sec-2-1-1}
\begin{itemize}
\item Convolution with Gaussian Kernel
$$ (K_\sigma * f)(x) = \int_{\mathbb{R}}^2 K_\sigma(x-y)f(y)dy $$
standard deviation $\sigma$, original image f
\item Convolution is multiplication in frequency domain
\item Gaussian Kernel is gaussian in frequency domain $\rightarrow$ \textbf{low-pass} filter
\item with linear diffusion $\partial_t u = \Delta u$ exists \textbf{unique analytic solution}:
      $$ u(x,t) = \begin{cases}f(x) & t=0 \\ (K_{\sqrt{2t}}*f)(x) & t>0 \end{cases} $$
\item satifies \textbf{maximum-minimum principle} $\rightarrow u(x,t)$ has only values between
maximum and minimum of original image
\end{itemize}
\subsubsection{Numerical Aspects}
\label{sec-2-1-2}
\begin{itemize}
\item \textbf{FFT} because convolution is multiplication in frequency domain $\rightarrow$ 
comp. costs independent of $\sigma$: O(N log(N))
\item \textbf{Solution in spatial domain} sampling gaussian and truncate at multiple of $\sigma$,
costs O(N$\sigma$)
\item \textbf{Discretisations of diffusion equation} $\rightarrow$ less efficient for large times
\end{itemize}
\emph{TODO: How to get discrete Derivatives and boundary conditions}
\subsubsection{Scale-Space Concept}
\label{sec-2-1-3}
Featuers T exists at different scales, the best scale is often not known
$\rightarrow$ we consider hierarchy of Features T with gradually increased scales\\
    \textbf{Scale-Space:}
$$ \{T_tf | t \ge 0\} $$
with T as Transformation at scale t, image f \\
    \textbf{Requirements:}
\begin{itemize}
\item architectural properties: e.g. semi group $T_{t+s}f = T_t(T_s(f))\; \forall s,t \ge 0$
\item simplification properties: eg. causality, maximum-minium priciple
\item invariances: e.g translation and rotation invariance
\end{itemize}
\textbf{Gaussian Scale-Space:}
$$ T_tf := K_{\sqrt{2t}}*f $$
\subsection{Image Pyramids}
\label{sec-2-2}
\subsubsection{Gaussian Pyramid}
\label{sec-2-2-1}
\textbf{Goal:} images on serveral resolution levels $\rightarrow$ discrete version of
scale-space with downsampling\\
    \textbf{Properties:}
\begin{itemize}
\item damp high frequencies to avoid aliasing $\rightarrow$ low-pass, than sample with half
of the freqency
\item Lowpass: 
\begin{itemize}
\item Sampled Gaussian (linear diffusion)
\item box filter (1D approximation of Gaussian)
\end{itemize}
\item more efficient than discrete scale-space due to downsampling
\item optimal complexity: linear ?
\item Drawbacks:
\begin{itemize}
\item which downsampling operator
\item downsampling of sparse data
\end{itemize}
\end{itemize}

\subsubsection{Laplacian Pyramid}
\label{sec-2-2-2}
\textbf{Goal:} decompose image into different frequency bands in \emph{spatial domain} \\
    \textbf{Properties:}
\begin{itemize}
\item downsampling with gaussian low pass like in Gaussian Pyramids
\item \emph{difference} of subsequent levels is a \textbf{band-pass}
\item upsampling of later levels is needed $\rightarrow$ pixel doubling
\end{itemize}
\subsection{Derivative Filters}
\label{sec-2-3}
Strong change in grey values indicate edges $\rightarrow$ important image feature, but still
low-level vision.
\begin{itemize}
\item derivative is calculated with finite differences
\item increased stencil size increases order of consitency
\item First order derivative: $\partial_x \approx \frac{1}{2h} [-1,0,1]$
\begin{itemize}
\item \textbf{Sobel} operator
\item convolution of derivative with binomial kernel
       \[ \partial_x \approx \frac{1}{4} \begin{bmatrix} 1 \\ 2 \\ 
       1 \end{bmatrix} * \frac{1}{2h}[-1,0,1] = 
       \frac{1}{8h} \begin{bmatrix} -1 & 0 & 1 \\ -2 & 0 & 2 \\ -1 & 0 & 1 \end{bmatrix} \]
\item still consitency order 2
\end{itemize}
\item Second order derivative
\begin{itemize}
\item \textbf{Laplacian}
       \[ \Delta u_{i,j} = \frac{1}{h^2} \begin{bmatrix} 
          0&1&0 \\ 1&-4&1 \\ 0&1&0 \end{bmatrix} u_{i,j} -  
	  \frac{1}{12}h^2(\partial_{xxxx}u_{i,j}+\partial_{yyyy}u_{i,j}) + O(h^4) \]
\item first term is rotation invariant, second not
\end{itemize}
\end{itemize}

\subsubsection{Edge Detection}
\label{sec-2-3-1}
\begin{itemize}
\item Idea: convolve image with Gaussian and compute gradient magnitude using \textbf{Sobel}
     $|\nabla u | = \sqrt{(\partial_x u)^2+(\partial_y u)^2}$
\item Advantage:
\begin{itemize}
\item first order derivatives more robust to noise than second order
\end{itemize}
\item Disadvantage:
\begin{itemize}
\item additional threshold parameter for gradient magnitude (in addition to standard deviation)
\item edges are to thick or below threshold
\item no closed contours
\end{itemize}
\item Treshold can be chosen dependent on histogram
\end{itemize}
Canny Edge Detector:
\begin{enumerate}
\item Gaussian smoothing, gradient magnitude calculation, as well as oriention of gradient
\begin{itemize}
\item choose edges above threshold T$_{\text{1}}$
\end{itemize}
\item \textbf{Non-Maxima Suppresion} If there exists pixels in orthogonal direction 
of edge with larger gradient, delete current pixel
\item \textbf{Hysteresis Thresholding} choose second threshold $T_2>T_1$, choose 
all pixel above $T_2$ and add all adjecent neighbors with threshold above $T_1$
\end{enumerate}
Laplacian of Gaussian:
\begin{itemize}
\item Gaussian smoothing, than compute Laplacian
\item Edges are zero crossings, since Laplacian is second order derivative
\item Advantages:
\begin{itemize}
\item closed contours
\item no additional Treshold parameter
\end{itemize}
\item Disadvantages:
\begin{itemize}
\item also detects minima of first derivative
\item more sensitive to noise
\end{itemize}
\end{itemize}
\subsubsection{Corner Detection}
\label{sec-2-3-2}
\begin{itemize}
\item corners are more characteristic than edges, intersection of edges
\item calculated with \textbf{structure tensor}:
     \[ J_\sigma = K_\sigma * (\nabla f \nabla f^T) = \begin{pmatrix}
       K_\sigma*f_x^2 & K_\sigma f_xf_y \\
       K_\sigma*f_xf_y & K_\sigma f_y^2 \end{pmatrix}
     \]
size of local neighbourhood $\sigma$ (also standard deviation of gaussian), * convolution
\item Eigenvalues $\lambda_1, \lambda_2$ of J$_{\sigma}$ gives local structure
\begin{itemize}
\item \(\lambda_1, \lambda_2 \approx 0 \rightarrow \text{ flat area}\)
\item \(\lambda_1 >> 0, \lambda_2 \approx 0 \rightarrow \text{ edge}\)
\item \(\lambda_1, \lambda_2 >> 0 \rightarrow \text{corner} \)
\end{itemize}
\item less robust than Edge detection
\end{itemize}
\subsubsection{Extension to color images}
\label{sec-2-3-3}
\begin{itemize}
\item calculation of gradient magnitude for each color channel seperatly makes no sense
\item Edge detection:
\begin{itemize}
\item \textbf{joint color gradient}:
\[ |\nabla f| = \sqrt{\sum_{i}|\nabla f_i|^2} \]
for channels i
\end{itemize}
\item Corner detection:
\begin{itemize}
\item \textbf{joint color structure tensor}:
\[ \sum_{i} J_{i,\sigma} = \sum_{i} \nabla f_i \nabla f_i^T \]
for channels i
\end{itemize}
\end{itemize}
\subsection{Hough Transform}
\label{sec-2-4}
\begin{itemize}
\item detect simple geometric objects, with few parameters (radius, position, \ldots{})
\item object boundaries satisfies $g(x,y, p_1,...,p_m) = 0$ with $p_i$ parameter
\begin{itemize}
\item Line: \( x cos(\phi) +y sin(\phi) -d = 0 \rightarrow \phi, d\) are parameter
\item Circle: \( |x-a|^2 + |y-b|^2 -r^2 = 0 \rightarrow \) center (a,b) and 
radius r are parameter
\end{itemize}
\item first select all points where gradient magnitude is larger than threshold
\item Every remaining pixel votes for all objects it is part of
\item The moste relevant parameter combination has most votes (\textbf{The majority rules})
\item Advantages:
\begin{itemize}
\item no fully connected contours neccessary
\item extremly robust
\end{itemize}
\item Disadvantage:
\begin{itemize}
\item memory and computional costs increase fast with number of parameters
       $\rightarrow$ pyramid like approach
\end{itemize}
\end{itemize}
\subsection{Moment Invariants}
\label{sec-2-5}
\textbf{Goal:} check if two objects from two images are identical, while the second
object may have been applied to certain transformations (scale, translations, rotation, etc)
\begin{itemize}
\item Moment of order p+q:
     \[ m_{p,q}:= \int_{-\infty}^{\infty}\int_{-\infty}^\infty x^p y^q f(x,y)dxdy = 
        \sum_i \sum_j i^p j^q f_{i,j} \]
\item \textbf{Uniqueness Theorem for Moments} if f is continous and 0 outside a bounded subdomain
all moments exist $\rightarrow$ reverse also holds if moments satisfy certain growth
condition
\item \textbf{Central Moments} are translation invariant:
\[ \mu_{p,q} := \int_{-\infty}^\infty \int_{-\infty}^\infty 
        (x-\bar{x})^p (y-\bar{y})^q f(x,y) dx dy = 
	\sum_i \sum_j (i-\bar{i})^p (j-\bar{j})^q f_{i,j}\]
with $\bar{x}=\frac{m_{1,0}}{m_{0,0}}$ and $\bar{y}=\frac{m_{0,1}}{m_{0,0}}$
as \textbf{center of gravity}
\item \textbf{normalized central moments} are in \emph{addition} scale invariant
\[ \eta_{p,q} := \frac{\mu_{p,q}}{(\mu_{0,0})^\gamma} \]
with $\gamma = \frac{p+q}{2}+1$ and $p+q \ge 2$
\item \textbf{seven moment invariants of Hu} are also rotation invariant (the first 6 are also
invariant under mirroring, the 7th changes sign)
\item also there exists invariant moments for \textbf{shearing} or all afine transformations
\item Hu moments 3,4,5,7 are also invariant under motion blur
\item Moments are \textbf{not exactly} invariant due to \emph{discretization}
\end{itemize}
\subsection{Photometric Invariants}
\label{sec-2-6}
one can also construct photometric invariants which are not influenced by certain
lightning conditions, namely:
\begin{itemize}
\item Global multiplicative changes
\item local multiplicative changes (shadows)
\item local additive changes (highlights, specular reflections)
\end{itemize}
Examples:
\begin{itemize}
\item \textbf{Log-Derivative Transform}:
\[ f \rightarrow \left(ln(f)_x, ln(f)_y \right)^T \]
invariant under global multiplicative changes
\item \textbf{Chromatic Space}
\[ (R,G,B) \rightarrow \left( \frac{R}{N},\frac{G}{N},\frac{B}{N} \right) \]
normalization $N=\frac{1}{3}(R+G+B)$, invariant under global and
local multiplicative changes
\item \textbf{HSV Color Space}: Hue component is invariant under global and local multiplicative
changes as well as local additiv changes
\end{itemize}
\subsection{Texture Analysis}
\label{sec-2-7}
\begin{itemize}
\item no formal definition
\item local stochastic fluctations: size, shape, color, orientation etc may vary
\item often empircally successful statistic expressions for analysis
\item edge detectors not useful to seperate regions with different textures\\
\end{itemize}
\textbf{Goal}: describe complex structure with few parameters

\subsubsection{Texture Analysis without neighborhood context}
\label{sec-2-7-1}
\begin{itemize}
\item \textbf{central moment of histogram} (statistical moments) with order k:
\[ M_k := \frac{1}{N}\sum_{i=1}^N(f_i-\mu)^k =
         \sum_{i=1}^L(z_i - \mu)^k p(z_i)\]
with N pixels, grey values $f_i = z_i$, mean $\mu$ and probability for greyvalue $p(z_i)$
\item Important Examples:
\begin{itemize}
\item k=2 Variance
\item k=3 Skewness (Schiefe)
\item k=4 Kurtosis (Woelbung)
\end{itemize}
\item \textbf{Texture Discriminations}: replace greyvalues with texture attribute 
(e.g. statistical moment) within some window $\rightarrow$ apply 
classical edge detection on resulting image
\item Disadvantage:
\begin{itemize}
\item Statistical moments are \textbf{global} description for region
\item \textbf{spatial ordering} of pixel doesn't matter $\rightarrow$ checkerboard and
block with half white and half black have same statistical moments
\item texture descriptor without spatial information is called \textbf{first-order statistics}
\end{itemize}
\end{itemize}
\subsubsection{Texture Analysis with neighborhood context}
\label{sec-2-7-2}
\textbf{Possiblity 1: Multiscale Representation}
\begin{itemize}
\item characterize image on different scales with
\begin{itemize}
\item scale spaces or
\item pyramids
\end{itemize}
\item afterwards use first-order statistics texture descriptor
\item Example: ?
\end{itemize}
\textbf{Possiblity 2: Coocurrence Matrices}
\begin{itemize}
\item Historgram of neighborhood relations
\item displacement vector d specifies which points are compared
\item \textbf{bivarite grey value histogram} specifies relative frequency of greyvalue i with
greyvalue j in direction d
\item Example:
\includegraphics[width=10cm]{./occurence_matrix.png}
\item information in coocurence matrix is dense, has to be condensed:
\begin{itemize}
\item \textbf{Largest Probability}: maximal p(i,j) is called \emph{mode}
\item \textbf{Contrast}: $\sum_{i,j}(i-j)^2 p_{i,j} \rightarrow$ small if similar grey values
in neighborhood
\item \textbf{Homogeneity}: $\sum_{i,j} \frac{p_{i,j}}{1+|i-j|} \rightarrow$ large if 
similar grey values in neighborhood
\item \textbf{Entropy}: $-\sum_{i,j}p_{i,j} log(p_{i,j}) \rightarrow$ mesaure of randomness and
is maximal if all possible configurations are equally likely
\end{itemize}
\end{itemize}
\textbf{Possibility 3: Fourier Domain}
\begin{itemize}
\item good for periodic textures
\item consider fourier spectrum in polar cooridnates $|f(r,\phi)|$
\item compute $g(r)$ by integrating over $\phi$ and $\h(\phi)$ by integrating over r
      $\rightarrow$ maximum, variance etc of these functions can discriminate textures
\end{itemize}
\subsection{Scale Invariant Feature Transform (SIFT)}
\label{sec-2-8}
\textbf{Goal:} detect \emph{feature points} and use descriptors to repesent them as unique as possible,
while maintaining \emph{invariance under scaling, translation and rotation}
\begin{enumerate}
\item \textbf{Difference of Gaussians (DoG):}
\begin{itemize}
\item Consider Gaussian scale-space $K_\sigma * f$ for multiples of $\sigma$
\item Calculate difference between consecutive levels of scale-space
        $\rightarrow$ \emph{band-pass} filter
\item Relation to \emph{Laplacian of Gaussian} (LoG):
\begin{itemize}
\item $\frac{\partial K_\sigma}{\partial \sigma} = \sigma \Delta K_\sigma
	  \approx \frac{K_{k\sigma}-K_\sigma}{k\sigma -\sigma}$
\item $K_{k\sigma}-K_\sigma \approx (k-1)\sigma^2 \Delta K_\sigma
	  \approx \sigma^2 \Delta K_\sigma$
\item $\sigma$$^{\text{2}}$ is needed for true scale invariance
\item is called \emph{normalized LoG}
\item normalized LoG has extremal value at \emph{characteristic scale} of feature
\end{itemize}
\item find extrema by comparing values of DoG in neigborhood of location and scale
\end{itemize}
\item \textbf{Taylor approximation for sub pixel accurency}
\begin{itemize}
\item at found extrema perform second order Taylor expansion:
        \[ D(x_i+h)\approx D(x_i)+\nabla D(x_i)^T h + \frac{1}{2}h^T \mathcal{H}(D(x_i))h \]
\item new extrema is found by setting derivative to zero and solving the resulting equation
for $h$
\item if $h>0.5 \rightarrow$ repeat process for closest interger location to new found extrema
\item remove feature point if
\begin{itemize}
\item extremal value is not large enough (low contrast)
\item if principal curvatures is too large (edge instead of corner)
\item principal curvature is proportional to eigenvalues of Hessian \mathcal{H}
$\rightarrow$ ratio of eigenvalues can be computed efficenlty by trace and
determinant of hessian
\end{itemize}
\end{itemize}
\item \textbf{Histogram of Gradients (HoG):}
\begin{itemize}
\item compute the gradient for every pixel in a circular neighborhood of a feature point
at corresponding scale
\item gradient is weighted with gaussian ($1.5\sigma$) centered at location of feature point
\item generate histogram of orientation with 10 degree steps and weight each entry with
magnitude of gradient
\begin{itemize}
\item largest entry is dominant orientation
\item for every entry with >80\% in histogram $\rightarrow$ create additional feature point
\end{itemize}
\end{itemize}
\item \textbf{SIFT Key point Descriptor:}
\begin{itemize}
\item for each feature point consider neighborhood of 16x16 pixels, divided into 16 blocks
of 4x4 pixels
\item for each 4x4 block calculate HoG with 45 degree steps (8 directions)
(compensated for dominant direction and location?)
\item store all 8 values of HoG for all 16 blocks in vector (128 entries)
\item normalize it
\end{itemize}
\end{enumerate}
\textbf{Invariance to scaling, shift and rotation:} neighborhood information computed at
caracteristic scale relative to feature location and compensated by dominant orientation\\
   \textbf{Invariance to illumination:} gradient is invariant to additive changes, final normlization
of descriptor vector makes it invariant to multiplicative changes

\section{Image Sequence Analysis I}
\label{sec-3}
\subsection{Local Methods}
\label{sec-3-1}
\textbf{Goal:} Given a image sequence $f(x,y,z)$ with time variable z, calculate displacment
vector $(u(x,y,z),v(x,y,z))^T$ called \textbf{optical flow}\\
   Examples:
\begin{itemize}
\item reconstruction of 3D-World from image sequence (structure-from-motion)
\item deformation in material science
\item video processing / video coding
\item interlacing in cameras
\end{itemize}

\subsubsection{linearised optical flow constraint (OFC)}
\label{sec-3-1-1}
Properties:
\begin{itemize}
\item assume greyvalue does not change in consecutive images, therefore
      \[ f(x+u,y+v,z+1) = f(x,y,z) \]
\item linearisation with Taylor for small u,v gives \emph{linearized optical flow constraint:}
      \[ 0=f(x+u,y+v,z+1)-f(x,y,z) = f_x(x,y,z)u+f_y(x,y,z)v+f_z(x,y,z)= 
         f_xu+f_yv+f_z\]
\item grey value constancy is often correct, illumination changes often very slowly
over several frames
\item linearisation assumption is violated more frequently
\begin{itemize}
\item either use OFC without linearisation (more difficult models) or
\item use coarse-to-fine approaches with pyramids
\end{itemize}
\item OFC don't has unique solution (2 variables, 1 equation):
\begin{itemize}
\item OFC specifies parallel flow component to gradient
        \[ 0= f_xu+f_yv+f_z=(u,v)\nabla f +f_z \]
\item orthogonal flows to $\nabla f$ can be added without violating OFC
\item \textbf{aperture problem}
\item more constraints are needed
\end{itemize}
\item Normal flow (normal component of flow vector $(u,v)$):
\begin{itemize}
\item with $(u,v)\nabla f = -f_z$ can be caluclated (in contrast to tangential component)
\[ (u_n,v_n) = (u,v)...  \]
what ever\ldots{} TODO!!!
\item Normal flow gives poor results
\end{itemize}
\end{itemize}

\subsubsection{Spatial approach of Lucas and Kanade}
\label{sec-3-1-2}
\textbf{Model:}
\begin{itemize}
\item optical flow at specfic point is \textbf{constant vector} $(u,v)$ in 
ballshaped neighborhood $B_r(x,y)$
\item this flow vector minimizes the local energy
      \[ E(u,v) = \frac{1}{2}\int_{B_r(x,y)} (f_xu+f_yv+f_z)^2 dxdy \]
\item solution is vector which fullfills OFC best in least square sense in the
neighborhood $B_r(x,y)$
\item setting partial derivatives to zero yields 2x2 LGS:
      \[ \begin{pmatrix} \int_{B_r}f_x^2 dxdy & \int_{B_r}f_xf_y dxdy \\ 
      \int_{B_r}f_xf_y dxdy & \int_{B_r}f_y^2 dxdy \end{pmatrix}  \begin{pmatrix} 
      u \\ v \end{pmatrix} 
      = \begin{pmatrix} -\int_{B_r}f_x f_z dxdy \\ -\int_{B_r}f_y f_z dxdy \end{pmatrix} \]
\end{itemize}
\textbf{Practical considerations:}
\begin{itemize}
\item often one replaces "hard" window $B_r$ with Gaussian convolution $\rightarrow$
fullfilling OFC in distance is not that important
\item Matrix in LGS is \emph{structure tensor} $J_r$
\item Problems:
\begin{itemize}
\item $rank(J_r) = 0$ both eigenvalues are zero, means \textbf{flat area}
\item $rank(J_r) = 1$ one eigenvalue is zero, means same gradient in whole area
        $\rightarrow$ infinitly many solutions, aperture problem persits
        $\rightarrow$ one can only compute \textbf{normal flow}
\end{itemize}
\end{itemize}
\textbf{Advantages:}
\begin{itemize}
\item simple and fast
\item often good results for costs
\end{itemize}
\textbf{Disadvantages:}
\begin{itemize}
\item problems if local constant flow vector assumptions is violated (e.g rotations)
\item local method $\rightarrow$ does not compute flow field at all locations
\end{itemize}

\subsubsection{Spatiotemporal approach of Lucas and Kanade}
\label{sec-3-1-3}
\begin{itemize}
\item replace spatial local constancy with local spatiotemporal constancy
(flow not only constant for one timestep, but for several?)
\item same LGS, but with Gaussian convolution over entire sequence
\item more robust
\end{itemize}

\subsubsection{Spatiotemporal approachh of Bigün}
\label{sec-3-1-4}
\textbf{Model:}
\begin{itemize}
\item Optical flow is regarded as orientation in space-time domain
\item we search for direction with least grey value change in neighborhood
      $B_r(x,y,z)$ (time variable z is now also considered)
\item Minimizing Energy for vector $w=(w_1,w_2,w_3)^T$:
      \[ E(w_1,w_2,w_3)= \int_{B_r(x,y,z)}(f_x w_1 + f_y w_2 + f_z w_3)^2 dx dy dz \]
\item Afterwards normalize $w_3$ to get spatial optical flow:
      \[ u=\frac{w_1}{w_3}, v=\frac{w_2}{w_3} \]
\item In contrast to Lucas-Kanade Method, we also optimize time component (\emph{total least square})
\end{itemize}
\textbf{Minimization:}
\begin{itemize}
\item desired vector $w$ is normalized eigenvector to smallest eigenvalue of:
      \[ \int_{B_r} \nabla_3 f \nabla_3 f^T dx dy dz \]
\item if integral is replaced with gaussian convolution ($K_r *(\nabla_3 f \nabla_3 f^T)$)
one gets 3D structure tensor $J_r$
\end{itemize}
\textbf{Flow classification:}
\begin{itemize}
\item $rank(J_r) = 0$, all eigenvalues are zero, no information about flow
\item $rank(J_r) = 1$, two low contrast eigenvalues, no unique flow, \textbf{aperture problem}
      $\rightarrow$ only normal flow can be calculated
\item $rank(J_r) = 2$, one vanishing eigenvalue, optical flow is eigenvector of smallest eigenvalue
normilized on the third component (as described above)
\item $rank(J_r) = 3$, assumption of locally constant flow is violated, either
      \textbf{flow discontinuity} or noise dominates
\end{itemize}
\textbf{Advantages:}
\begin{itemize}
\item high robustness to noise
\item provides detailed information on optical flow because of spatiotemporal structure tensor
\end{itemize}
\textbf{Disadvantages:}
\begin{itemize}
\item more complicated than Lucas-Kanada (numerical principal component analysis of 3x3 matrix)
\item also problems at flow discontinuites and non-translatory motion (e.g. rotation)
\item also no global method
\item more threshold parameters
\end{itemize}

\section{Image Sequence Analysis II}
\label{sec-4}
\subsection{Affine Lucas and Kanade Method}
\label{sec-4-1}
\begin{itemize}
\item instead of assuming local constant flow, we assume now \textbf{locally affine} flow
\item we consider affine parametrization of flow:
     \[ w = (u,v,1)^T = \begin{pmatrix} ax+by+c \\ dx+ey+f \\ 1 \end{pmatrix}
          = \begin{pmatrix} x&y&1&0&0&0&0 \\ 0&0&0&x&y&1&0 \\ 0&0&0&0&0&0&1 
        \end{pmatrix} \begin{pmatrix} a\\b\\c\\d\\e\\f\\1 \end{pmatrix} = M\cdot p\]
\item OFC can be rewritten as:
     \[ f_xu+f_yv+f_z = \nabla_3 f^T w = \nabla_3 f^T (Mp) = (M^T\nabla_3f)^Tp = r^Tp \]
\item locally affine model means \textbf{locally constant set of affine parameters}
\item local energy:
\[ E(p) = \int_{B_r}(p^Tr)^2 dxdy = p^T\int_{B_r}(rr^T dx dy)p = p^T J_r p  \]
with $J_r$ 7x7 affine equivalent of saptiotemporal structure tensor
\item Again setting partial derivative to zero yields 6x6 LGS
\end{itemize}
\subsection{KLT feature tracker}
\label{sec-4-2}
\textbf{Overview:}
\begin{enumerate}
\item feature selection
\item inter frame displacement for each feature (spatial approach from Luca and Kanade)
\item find new features at new places, discard features, which couln't be tracked
\item goto 2
\end{enumerate}
KLT tracker dont use feature descriptor (as does SIFT), only compares windows 
around features (corners), but does that efficently, because displacement
of feature is computed via optical flow

\subsubsection{Inter-frame displacement estimation}
\label{sec-4-2-1}
\begin{itemize}
\item uses spatial approach from Lucas and Kanade
\item motion between subsequent frames is small $\rightarrow$ linearisation is valid
\item only 2 unknows have to be solved (instead of 6 with the affine model)
\item since computed displacement is not perfect, use \textbf{iterative approach:}
\begin{enumerate}
\item \textbf{Start:} compute flow between subsequent frames
\item \textbf{Resgistration:} compensate second image: $f*(x,y,z+1):= f(x+u^k,y+v^k, z+1)$
\item \textbf{Correction}: compute flow $(du,dv)$ between $f$ and $f*$ 
(remaining flow after compensation)
\item \textbf{Update:} add correction to flow: $(u^{k+1},v^{k+1}) = (u^k, v^k)+(du,dv)$
\item \textbf{Loop:} goto 2 until max steps of corrections
\end{enumerate}
\item iterative approach needs \emph{sub-pixel values} $\rightarrow$ use \textbf{bilinear interpolation}
\end{itemize}
\subsubsection{Feature selection}
\label{sec-4-2-2}
\begin{itemize}
\item since we perform translational (spatial approach) Lucas/Kanade $\rightarrow$
we solve 2x2 LGS with spatial structure tensor as system matrix
\item reliable solution exists if:
\begin{itemize}
\item above noise level $\rightarrow$ both eigenvalues are large
\item well coditioned $\rightarrow$ eigenvalue do not differ by orders of magnitude
\end{itemize}
\item in practice $\lambda_2 > treshold$
\item comes down to \textbf{corner detection} (because system matrix is structure tensor)
\item neighborhood size is 3 times standard deviation (99.7\% of information)
\item only one feature in neighborhood
\end{itemize}
\subsubsection{Feature monitoring}
\label{sec-4-2-3}
\begin{itemize}
\item feature is discarded if cumulative error between current and first frame of the
neighborhood becomes too large
\[ e = \int_{B_r} (f(x+u,y+v,t)-f(x,y,1))^2 dx dy \]
\item complete motion may be complex, simple translation is not sufficent
      $\rightarrow$ use affine method of Lucas and Kanade to compute cumulative errors only
\end{itemize}
\subsection{Feature matching}
\label{sec-4-3}
\begin{itemize}
\item Match features points in different images (\textbf{not} neccessary in subsequent frames!)
\item therefore \textbf{feature descriptor} for every feature point is computed (e.g SIFT)
\item compare all feature descriptors in both images
     \[ diff(a,b) = \|a-b\|^2 \]
\item for each feature point in original image get best match in second image
\end{itemize}
\textbf{Advantages:}
\begin{itemize}
\item invariance properties of feature points $\rightarrow$ more accurate results
\item estimate global motion model from few parameters (epipolar geometry needs 7)
\end{itemize}
\textbf{Disadvantages:}
\begin{itemize}
\item yields non-dense flow fields not useful for many tasks
\end{itemize}

\section{Image Sequence Analysis III}
\label{sec-5}
\subsection{Variational methods}
\label{sec-5-1}
\textbf{Problems of local methods:}
\begin{itemize}
\item non-dense flow fields (at low gradient nothing can be said)
\item local constancy of optic flow not suitable for non-translatory motion
\end{itemize}
\textbf{Variational Methods:}
\begin{itemize}
\item also assume grey value constancy (yields OFC)
\item to solve aparture problem, second contraint: \textbf{smoothness} of flow field
     \[ \int_\Omega (|\nabla u|^2 + |\nabla v|^2) dx dy \quad \text{ is small} \]
\item smoothness constraint holds \textbf{globally} (in contrast to local constancy contraint)
\item can be relaxed to piecewise smoothness $\rightarrow$ requires nonquadratic penaliser
\end{itemize}

\subsubsection{Method of Horn and Schnuck}
\label{sec-5-1-1}
\begin{itemize}
\item find function $(u(x,y),v(x,y))^T$ which minimizes the energy functional
      \[ E(u,v) = \int_\Omega ((f_xu+f_yv+f_z)^2 + \alpha (|\nabla u|^2 + |\nabla v|^2)) dx dy \]
\item regularisation parameter $\alpha > 0$ determines smoothnes of flow field
\end{itemize}
\textbf{Advantages:}
\begin{itemize}
\item also suitable for non-translatory motions
\item transparent and flexible model: no hidden assumptions, assumptions can be modified
\item dense flow field due to \textbf{filling-in effect} $\rightarrow$ smoothness dominates at
locations where $|nabla f| \approx 0$, therefore propagation from the neighborhood
\end{itemize}
\textbf{Disadvantages:}
\begin{itemize}
\item only suitable for small displacement (with linearized OFC)
\item simple algorithms are slow
\end{itemize}
\subsection{A simple algorithm}
\label{sec-5-2}
\begin{enumerate}
\item Going to the Euler-Lagrange Equations
\begin{itemize}
\item Energy function
        \[ E(u,v) = \int_\Omega F(x,y,u,v, u_x,u_y,v_x,v_y) dxdy \]
\item satisfies Euler-Lagrange Equations
\begin{align*} \partial_x F_{u_x} + \partial_y F_{u_y} - F_u &= 0 \\
	   \partial_x F_{v_x} + \partial_y F_{v_y} - F_v &= 0 \end{align*}
\end{itemize}
\item Discretization:
\begin{itemize}
\item approximate derivates and laplacian by difference operators
\item yields 2 difference equations for each pixel
\item sparse but huge system of equations
\end{itemize}
\item Solving linear system:
\begin{itemize}
\item iterative method for example Jacobi method
\begin{itemize}
\item solve $Ax=b$
\item divide $A=D-N$  into diagonal matrix $D$ and rest $N$
\item Problem $Dx = Nx +b$ is solveable by fixed point iteration 
          $x_{k+1} = D^{-1}(Nx_k+b)$
\end{itemize}
\end{itemize}
\item Iteration step:
\begin{itemize}
\item iterate until stopping criteria (residium in Jacobi method)
\end{itemize}
\end{enumerate}
\subsection{Extensions and Generalisations}
\label{sec-5-3}
\begin{itemize}
\item Data term:
\begin{itemize}
\item other constraints (e.g sever illuminiation changes)
\item higher order Taylor approximation
\end{itemize}
\item Smoothness term:
\begin{itemize}
\item spatiotemporal smoothness instead of spatial one
\end{itemize}
\item Algorithm:
\begin{itemize}
\item faster iterative solver (e.g SOR)
\item coarse-to-fine warping with pyramid
\end{itemize}
\end{itemize}

\section{3D Reconstruction}
\label{sec-6}
\subsection{Camera Geometry}
\label{sec-6-1}
\begin{itemize}
\item Relation between Point $(X,Y,Z)$ and the mapping $(x,y)$ onto the virtual image plane
\[ \frac{x}{X} = \frac{y}{Y} = \frac{f}{Z} \]
with $f$ focal length
\end{itemize}

\subsubsection{Homogenous coordinates}
\label{sec-6-1-1}
\begin{itemize}
\item since the relation is nonlinear, one uses \textbf{homogenous coordinates}
\begin{itemize}
\item additional coordinate as \emph{scaling factor}
       \[ \begin{pmatrix} x\\y \end{pmatrix}
          \rightarrow \begin{pmatrix} wx \\wy \\ w \end{pmatrix} \]
\item yields projection matrix
       \[ \begin{pmatrix} wx \\wy \\ w \end{pmatrix}
          = \begin{pmatrix} f&0&0&0 \\ 0&f&0&0 \\ 0&0&1&0 
            \end{pmatrix}\begin{pmatrix} X\\Y\\Z\\1 \end{pmatrix} \]
\end{itemize}
\end{itemize}
\subsubsection{Extrinsic camera parameters}
\label{sec-6-1-2}
\begin{itemize}
\item describe rotataion and  translation of camera in world frame
\item are described in homogenous coordinates (rotation/translation matrix)
\item 6 parameters (3 translation, 3 rotations)
\end{itemize}
\subsubsection{Intrinsic camera parameters}
\label{sec-6-1-3}
\begin{itemize}
\item describe image plane inside the camera
\item 5 parameters:
\begin{itemize}
\item origin of image plane (2)
\item pixels size in each direction (2)
\item angle of cooridnate axis (1)
\end{itemize}
\end{itemize}
\subsubsection{Camera calibration}
\label{sec-6-1-4}
\begin{itemize}
\item full projection matrix is retrived by
      \[ M_{intrinsic} \cdot P \cdot M_{extrinsic} \]
\item Many algorithms to find parameters in projection matrix
\item Basic Idea:
\begin{itemize}
\item find corresponding points of known object and its image
\item one point gives 2 contraints $\rightarrow$ 6 points are at least needed
\item using more with least square yields less senstivity to errors
\end{itemize}
\item also consider model for \textbf{lens distortion}
\end{itemize}
\subsubsection{Projective Mappings}
\label{sec-6-1-5}
\begin{itemize}
\item to get 2D coordinates divide by last coodinate of homogenous coordinates
      $\rightarrow$ nonlinear transformation
\begin{itemize}
\item parrallel lines in 3D are intersection in 2D
\item center of gravity in 3D is \textbf{not} mapped to c.o.g in 2D
\item for $w \rightarrow 0$ small pertubations in 3D cann lead to large deviations in 2D
\end{itemize}
\item \textbf{Affine mapping}:
\begin{itemize}
\item linear approximation, good for small variation in depth (e.g. satellite images)
\item $w$ is approximated with constant $c$
\item no bad properties of nonlinear mapping, only 6 degrees of freedom
\end{itemize}
\item \textbf{Orthogonal mapping:}
\begin{itemize}
\item complete ignores depth information
\item to crude for most applications, but applicable for some (e.g. shape from shading)
\end{itemize}
\end{itemize}

\subsection{Epipolar Geometry, Stereo Matching}
\label{sec-6-2}
\begin{itemize}
\item \textbf{baseline:} line connecting both optical centers (\textbf{length} $b$)
\item \textbf{conjugated points:} two points in the two images from same 3D point
\item \textbf{epipolar plane:} plane connecting point in 3D and both optical centers
\item \textbf{epipolar lines:} intersection of epipolar plane and iamge plane
\item \textbf{disparity:} difference of conjugated points
\end{itemize}

\subsubsection{Orthoparallel setup}
\label{sec-6-2-1}
\begin{itemize}
\item two cameras with parallel optical axis
\item epipolar lines are parrallel
\item depth calculation with conjugates points $x_1,x_2$:
      \[ z = \frac{bf}{x_1 - x_2} \]
\item \textbf{Problem:} disparity is often hard to measure, only pixel size distance
\end{itemize}
\subsubsection{Converging camera setup}
\label{sec-6-2-2}
\begin{itemize}
\item epipolar lines are not neccessary parrallel anymore
\item conjugated points must satisfy \textbf{epipolar constraint}:
\[ \tilde{x_1}^T F \tilde{x_2} = 0 \]
with $\tilde{x_1}$ in homogenous coordinates, $F$ 3x3 fundamental matrix
\item $F$ has rank 2 $\rightarrow$ 7 degrees of freedom
\item not possible to extract intrinsic and extrinsic parameters from $F$
\item if $F$ is known, one can find epipolar line in other image via
      \[ \tilde{x_1}^Tl_1 \quad \text{with}\quad l_1 = Fx_2 \]
\item creates reduced 1D search space for stereo matching
\end{itemize}
\subsubsection{Estimation of fundamental matrix}
\label{sec-6-2-3}
\begin{itemize}
\item epipolar constraint can be rewritten into form with vector $s$ containing all coordiantes
of the conjugated points and vector $f$, containing the 9 entries of the $F$ matrix
\[ 0=\tilde{x_2}^T F \tilde{x_1} = s^Tf \]
\item at least 8 conjugated points are needed to get entries of f
\item solve LGS
      \[ Sf = 0 \quad \text{with} \quad S=\begin{pmatrix}s_1^T \\ \vdots \\ s_N^T \end{pmatrix} \]
\item Problems:
\begin{itemize}
\item $f=0$ always works
\item $N>8$ no solution, over-determined system
\item $N<8$ or linear dependent constraints leads to infinitely many solutions
\end{itemize}
\item Smallest sum of squares to get f:
      \[ \min_f \sum_{i=1}^N (s_i^tf)^2 \quad \text{with}\quad |f|=1 \]
\item since F has rank 2, we enforce this by setting the smallest singular value in SVD to 0
(rank 2 projection?)
\end{itemize}
\subsubsection{Stereo reconstruction as correspondence problem}
\label{sec-6-2-4}
\begin{itemize}
\item finding conjugated points is \textbf{correspondence problem} similar to optical flow,
but with large displacements
\item but epipolar constraint $\rightarrow$ smaller search space (1D)
\item \textbf{Correlation-Based methods (local method):}
\begin{itemize}
\item epipolar constraint $\rightarrow$ 1D search space
\item limited maximum displacement
\item neighborhood is similiar $\rightarrow$ computed via \textbf{correlation coefficient}
\item simple and fast, but unreliable in flat regions $\rightarrow$ structured illumination
(project grid on object)
\end{itemize}
\item \textbf{Variational methods (global method):}
\begin{itemize}
\item similar to variational methods in optical flow computation
\item minimize energy functional (here only disparites/displacement in x direction):
        \[ E(u) = \int_\Omega ((f_2(x+u,y)-f_1(x,y))^2 + \alpha |\nabla u|^2)dx dy \]
\item data term $(f_2 - f_1)^2$ may not be convex $\rightarrow$ local minima
        $\rightarrow$ no global convergence for numerical methods
\item Possible solution: use gaussian smoothing to reduce local minima
(\textbf{gradually reduce gaussian} and use previous solution as initialization)
\item again use Euler-Lagrange equations and iterative algorithm to solve resulting
system of nonlinear equations
\item Advantage: \textbf{filling-in effect} gives dense disparity maps
\item Disadvantage: high computional effort
\end{itemize}
\end{itemize}
\subsection{Shape from shading}
\label{sec-6-3}
\begin{itemize}
\item get shape from single illuminated photograph, with know reflectance properites and
light source position
\item important in astronomy $\rightarrow$ no stero reconstruction possible (too far away)
\end{itemize}

\subsubsection{Illumination and Reflectance}
\label{sec-6-3-1}
\begin{itemize}
\item radiation from surface is dependent on
\begin{itemize}
\item illumination direction
\item reflectance properties
\end{itemize}
\item BRDF describes ratio between incoming light and outgoing light (dependent on angle)
\item special cases
\begin{itemize}
\item ideally reflecting surface: reflects incoming ray in only one direction (mirror)
\item Lambertian Surface: light is reflected in \textbf{all} directions, \textbf{albedo} $\rho$
describes the ratio of reflected light
\end{itemize}
\end{itemize}
\subsubsection{Basic Assumptions}
\label{sec-6-3-2}
\begin{itemize}
\item \textbf{Lamberitan surface:}
\begin{itemize}
\item reflected radiance $R$ is proportional ($\rho$) to the cosine between light direction $s$
and surface normal $n$:
\[ R(X,Y,Z) = \rho s^Tn \]
\item we assume $s$ and $p$ are known
\end{itemize}
\item \textbf{Orthographic camera model}:
\begin{itemize}
\item image coordinates $(x,y)$ can be directly identified with world coordinates $(X,Y)$
\item reflected radiance $R$ corresponds to greyvalue of image (\textbf{image irradiance equation})
        \[ f(x,y) = R(X,Y,Z) \]
\end{itemize}
\item \textbf{Depth reconstruction:}
\begin{itemize}
\item consider partial derivative of depth $Z$: $p=\partial_x Z, q=\partial_y Z$
\item differenting the surface wrt. x and z gives two tangential vectors
        \[ (1,0,p)^T, (0,1,q)^T \]
\item outer product gives normal vector on surface (normalized):
        \[ \frac{1}{\sqrt{1+p^2+q^2}} \begin{pmatrix} -p \\ -q \\ 1 \end{pmatrix} \]
\item relation between grey values and partial derivates:
        \[ f(x,y) = R(X,Y,Z) = \rho s^T n = \frac{\rho}{\sqrt{1+p^2+q^2}} 
	   s^T \begin{pmatrix} -p \\ -q \\ 1 \end{pmatrix} \]
\end{itemize}
\end{itemize}
\subsubsection{Variational method by Ikeuchi and Horn}
\label{sec-6-3-3}
\textbf{Assumption}: $p$ and $q$ vary smoothly in space
\begin{itemize}
\item Energy functional(\textbf{global method}):
      \[ E(p,q) = \int_\Omega ((R(p,q)-f(x,y))^2 + \alpha (|\nabla p|^2 + |\nabla q|^2))dx dy \]
\item As with stereo reconstruction, data term maybe not convex $\rightarrow$ multiple minima
\item use gaussian smoothing, gradually reduce scale
\item Again use Euler-Lagrange equation and solve resulting nonlinear difference equations
with iterative solver
\item Problem: $p$ and $q$ are not independent
\begin{itemize}
\item yields inconsitencies?
\item solution 1: additional consistency term $(p_y -q_x)^2$ in energy functional
\item solution 2: after each iteration project $p$ and $q$ to closest integrable pair
of functions $\rightarrow$ does not garantee stability
\item solution 3: use variational model for Z instead for $p$ and $q$ $\rightarrow$
more complicated data term
\end{itemize}
\end{itemize}
\subsubsection{Photometeric reconstruction}
\label{sec-6-3-4}
\begin{itemize}
\item photograhps of same object in same position, but \textbf{different illuminations}
\item new energy functional (for $m$ images):
      \[ E(p,q) = \int_\Omega \left( \sum_{i=1}^m (R_i(p,q)-f_i(x,y))^2 +
         \alpha (|\nabla p|^2 + |\nabla q |^2) \right) dx dy \]
\item basically same functional, but sum over all images
\item \textbf{Advantage:}
\begin{itemize}
\item no correspondece problem like in stero reconstruction
\item more robust and precise to single image shape-shape-from-shading, since
error are averaged
\end{itemize}
\end{itemize}

\section{Foundations}
\label{sec-7}
\subsection{Isotropic nonlinear diffusion}
\label{sec-7-1}
\textbf{Idea:}
\begin{itemize}
\item avoid drawbacks of linear diffusion: delocalization and blurring of edges
\item linear diffusion does \textbf{not} allow edge-enhancing
\item intraregional better than interregoinal smoothing
\item modifiy flux function with function $g(|\nabla u|^2)$ (was $-I\nabla u$ in linear case):
     \[ \partial_t u = \text{div}(g(|\nabla u|^2) \nabla u) \]
\item function $g$ is called \emph{diffusivity} and is decreasing with increasing $|\nabla u|^2$ ?
\end{itemize}

\subsubsection{Perona-Malik-filter:}
\label{sec-7-1-1}
\begin{itemize}
\item special $g$ function:
     \[ g(|\nabla u|^2) = \frac{1}{1+|\nabla u|^2/\lambda^2} \]
\item yields \emph{flux function} (multiplication with $\nabla u$)
     \[ \Phi(\nabla u) = \frac{\nabla u}{1+|\nabla u|^2/\lambda^2} \]
\item for 1D case this yields
     \[ \partial_t u = \text{div}(\Phi(u_x)) = \partial_x(\Phi(u_x)) = \Phi'(u_x)u_{xx} \]
\item therefore
\begin{itemize}
\item \textbf{foreward diffusion} (smoothing) for $\Phi'(u_x) >0 \rightarrow |u_x| < \lambda$
\item \textbf{backward diffusion} (edge-enhancing) for $\Phi'(u_x) <0 \rightarrow |u_x| > \lambda$
\end{itemize}
\item properties of $g$:
\begin{itemize}
\item negative exponential function
\item g(0)=1
\end{itemize}
\end{itemize}
\subsubsection{Regularised nonlinear diffusion}
\label{sec-7-1-2}
\textbf{Goal:}
\begin{itemize}
\item become more independent of the choice of discretization
\item reduce starcaising artifacts and problems with noise
\item \textbf{well posedness} ?
\end{itemize}
Use \textbf{gaussian smoothed image} for \emph{diffusivity}:
\begin{itemize}
\item more robust under noise
\item still able to enhance edges
\item mathematical results:
\begin{itemize}
\item well posedness: unique solution $u \in C^\infty$
\item preservation of average grey value
\item minimum-maximum principle: u is bounded by infimum and supremum of initial image $f$
\item for $t \rightarrow \infty$ converges $u$ to average grey value
\end{itemize}
\item numerical results:
\begin{itemize}
\item $g\leq 1$ yields explicit scheme if time stepsize $< 0.25$
\end{itemize}
\end{itemize}

\subsection{Anisotropic nonlinear diffusion}
\label{sec-7-2}
\begin{itemize}
\item isotropic nonlinear diffusion has problems with noisy edges
\item anisotropic nonlinear diffusion takes direction of edge into account
\item replaces scalar-valued diffusivity $g$ with \textbf{diffusion tensor D}
\end{itemize}

\subsubsection{Edge-enhancing anisotropic diffusion (EED)}
\label{sec-7-2-1}
\begin{itemize}
\item smooting within regions, while inhibiting diffusion orthogonal to edges
\item Choice of diffusion tensor:
\begin{itemize}
\item one eigenvector $v_1$ orthogonal to $\nabla u_\sigma$
\item one eigenvector $v_2$ parallel to $\nabla u_\sigma$
\item eigenvalue $\lambda$$_{\text{1}}$ is similar to diffusivity $g$ (orthogonal to edge)
\item eigenvalue $\lambda$$_{\text{2}}$ is 1 (along edge)
\item resulting diffusin tensor:
        \[ D(\nabla u_\sigma) = (v_1, v_2) \cdot diag(\lambda_1, \lambda_2)
	   \cdot \begin{pmatrix} v_1^T \\ v_2^T \end{pmatrix} \]
\end{itemize}
\item EED is not good for closing interrupted lines in flow like structures
\begin{itemize}
\item Gaussian convolution cancel adjacent gradients with same direction
but opposite orientation
\item smoothed flow field is of no use
\end{itemize}
\end{itemize}
\subsubsection{Coherence-enhancing anisotropic diffusion (CED)}
\label{sec-7-2-2}
\begin{itemize}
\item try to smooth along coherent structures (edges)
\item diffusion tensor is based on properties of structure tensor
\begin{itemize}
\item eigenvectors are the same as of structure tensor ($J_\rho(u_\sigma)$)
\item eigenvalue $\lambda_1 = \alpha >0$ small (orthogonal to edges)
\item eigenvalue $\lambda_2$ dependent on \textbf{coherence}
\item coherence: $(\mu_1 - \mu_2)^2$ (eigenvalues of structure tensor)
\end{itemize}
\item numerical properties:
\begin{itemize}
\item mixed derivatives
\item standard space discretization violate non-negativity (no convex combination?)
        $\rightarrow$ many important properties (e.g. minimum-maximum priciple) don't hold
\end{itemize}
\end{itemize}

\section{Segmentation}
\label{sec-8}
\subsection{Mumford-Shah model}
\label{sec-8-1}
\begin{itemize}
\item \emph{the} prototype of \textbf{energy based} segementation
\item energy function:
\[ E(u,K) = \int_\Omega (u-f)^2 dxdy + \alpha \int_{\Omega\setminus K} |\nabla u|^2
        dx dy + \lambda l(K) \]
with $u$ smoothed version of $f$, and K are edges
\item first term is called \textbf{similarity term} and penalises deviations from original image
\item second term is called \textbf{smoothness term}
\item third term is called \textbf{contour term} and penalises edge length ($l(K)$ is length of edges)
\item mathematical challencing due to free segementation bounderies
\end{itemize}

\subsubsection{Mumford-Shah cartoon model}
\label{sec-8-1-1}
\begin{itemize}
\item approximation, with $\alpha \rightarrow \infty$, therefore one assumes that $u$ is
      \textbf{constant} in each segement
\item energy function:
      \[ E(u,K) = \int_{\Omega\setminus K} (u-f)^2 dxdy +\lambda l(K) \]
\item \textbf{properties:}
\begin{itemize}
\item \emph{preservation of mean:} within one segement $f$ and $u$ have the same mean
\item \emph{existence of gobal minimiser:} global minimum exits, but also many local ones
\item \emph{regularity of boundaries:}  boundaries are at least $C^1$
\end{itemize}
\end{itemize}
\subsubsection{Approximation method}
\label{sec-8-1-2}
\begin{itemize}
\item segmentation is called \textbf{2-normal} if every new segmentation K' which results from
merging two segments of K has larger energy
\item restriction to 2-normal segmentation yields following properties:
\begin{itemize}
\item \textbf{limited number} of regions
\item \textbf{elimination of small regions:} area of each region is lower bounded by constant
\item \textbf{elimination of thin regions:} region boundary length is upper bounded
\item \textbf{smoothness of boudnaries:} all boundaries are (almost) everwhere $C^1$
\end{itemize}
\end{itemize}
\subsubsection{Algorithmic realisation}
\label{sec-8-1-3}
\begin{enumerate}
\item initialization: each pixel is one segment
\item for each pair of neighboring segments: 
\begin{itemize}
\item compute $\lambda$ for which merging would decrease energy
\item energy for merged segments can be calculated without evaluating whole functional
\item new energy is old + difference in greyvalus - length of merged boundaries
\end{itemize}
\item merge segments with smallest $\lambda$ value
\item repeat steps 2-3 until desired number of regions or specific value for $\lambda$ is reached
\end{enumerate}
\subsubsection{Ambrosio-Tortorelli model}
\label{sec-8-1-4}
WHAT EVER!!!!

\subsection{Continous scaled morphology}
\label{sec-8-2}
\subsubsection{Classical morphology}
\label{sec-8-2-1}
\begin{itemize}
\item process images by computing maxima (dilation) or minimum (erosion) in neighborhood
(structuring element)
\item basic morphological processes:
\begin{itemize}
\item dilation: $(f\oplus B)(x):= \sup \{f(x-y)|y \in B\}$
\item erosion: $(f\ominus B)(x):= \inf \{f(x+y)|y \in B\}$
\item opening: $(f\circ B)(x):= ((f\ominus B)\oplus B)(x)$
\item closing: $(f\bullet B)(x):= ((f\oplus B)\ominus B)(x)$
\end{itemize}
\item invariant under monotonically increasing grey level rescaling $\rightarrow$ constrast
does not matter
\item \textbf{not able to enhance edges}
\end{itemize}
\subsubsection{Mophological curve evolution}
\label{sec-8-2-2}
\begin{itemize}
\item consider binary image with boundary curve $c_0 = (x(p),y(p))^T$
\item dilate it with a structuring element
\begin{itemize}
\item e.g. disc with radius $t$ gives parital derivate of curve $c(t)$
\[ \partial_t c = n \]
with n normal vector
\item for abritary structuring elements:
\[ \partial_t c = \beta(\kappa) \cdot n \]
with speed function $\beta(\kappa)$
\end{itemize}
\item applying level set ideas, one embeds $c$ in a smooth image $u(x,t)$
\item differentiation wrt. $t$ and $\partial_t c = \beta(\kappa)n$ yields
      \[ \partial_t u = \beta(\kappa)|\nabla u| \]
\end{itemize}
\subsubsection{PDEs for continous-scale morphology}
\label{sec-8-2-3}
\begin{itemize}
\item consider disc as structuring elements
\item yields speed function $\beta(\kappa) = \pm 1$
\begin{itemize}
\item $+$ for dilation, $-$ for erosion
\end{itemize}
\item difference to diffusion:
\begin{itemize}
\item only first order derivatives
\item forward and backward evolution equally stable ?
\end{itemize}
\item \textbf{Theoretical results:}
\begin{itemize}
\item well-posedness
\item minimum-maximum principle
\item \emph{scale-space property} $\rightarrow$ reduce number of local  minima/maxima, while preserving
their location
\item but noise sensitivity
\end{itemize}
\item \textbf{Numerical methods:}
\begin{itemize}
\item normal difference fail quickly (why?)
\item use of \emph{upwind schemes} $\rightarrow$ one-sided differences
\item if derivative is positiv $\rightarrow$ use finte differece from point in positive direction,
otherwise negtive direction
\end{itemize}
\item \textbf{Advantages:}
\begin{itemize}
\item sub-pixel accurency
\item good results for non-digitally scalable structuring elements (elements which are
easy rescaleable?)
\end{itemize}
\item \textbf{Disadvantage:}
\begin{itemize}
\item slower
\item dissipative effects: shocks became blurred $\rightarrow$ use 
        \emph{flux-corrected transport schemes (FCT)} (yields sharp edges)
\end{itemize}
\end{itemize}

\subsubsection{Shock filtering}
\label{sec-8-2-4}
\begin{itemize}
\item morpholgical deblurring technique
\item dilation around maxima ($\Delta u < 0$)
\item erosion around minima ($\Delta u > 0$)
\item creates shocks around edge locations ($\Delta u = 0$)
\item noise sensitivity
\item image enhancment method
\item many forms:
\begin{itemize}
\item simplest: $\partial_t u = -\text{sign}(\Delta u) |\nabla u|$
\item typ 1 with $\eta || \nabla u$: $\partial_t u = -\text{sign}(u_{\eta\eta}) |\nabla u|$
\item or gaussian smoothed version of typ 1
\item \emph{coherence-enhancing shock filter}: replace $u_{\eta\eta}$ with normalized dominant
eigenvector of structure tensor
\end{itemize}
\end{itemize}
\subsection{Mean curvature motion}
\label{sec-8-3}
\begin{itemize}
\item curve evolutions has speed function $\beta(\kappa)$ which depends on curvature $\kappa$
\item until now $\kappa$ was ignored $\rightarrow$ speedfunction constant
\item MCM performs anisotropic smoothing along \textbf{isophotes} (orthogonal to gradient)
     \[ \partial_t u = \partial_{\xi\xi}u \quad \text{with} \quad \xi \perp \nabla u \]
\item other forms:
     \[ \partial_t u = \partial_{\xi\xi}u = |\nabla u| \text{div}\frac{\nabla u}{|\nabla u|} 
        =  \text{curv}(u) |\nabla u| \]
\item unlike diffusion filter does \textbf{not} preserve average grey value
\item depending on sign of curvature curv(u) acting as erosion and dilation
\item not first order hyperbolic PDE, it is second order parabolic PDE $\rightarrow$
unstable in backward time direction
\end{itemize}

\subsubsection{corresponding curve evolution}
\label{sec-8-3-1}
\begin{itemize}
\item evolution of a shape $c(p)$ is given by \textbf{geometric heat equation} or
      \textbf{Euclidean shortening flow}
\begin{align*}
  \partial_t c(p,t) &= \kappa(p,t)\cdot n(p,t) \\
  c(p,0) &= c_0(p)
\end{align*}
\item propagates isophotes in normal direction
\item Euclidean invariant diffusion of isophotes???
\end{itemize}
\subsubsection{Theoretical results}
\label{sec-8-3-2}
\textbf{curve evolution:}
\begin{itemize}
\item smooth solution for finite intervall
\item number of extrema and inflection points of curvature is non increasing
\item shapes become \textbf{convex}
\item shapes shrink to \textbf{circular point}
\item if shape is inside another, it stays there the whole evolution
\end{itemize}
\textbf{image evolution:}
\begin{itemize}
\item well posedness
\item minimum-maximum principle
\item does not create 2D extreme (unline diffusion filters)
\item no contrast enhancement (no edge enhancement)
\end{itemize}
\textbf{extensions:}
\begin{itemize}
\item in 3D topological changes may occur
\item can be expanded to vector valued images $\rightarrow$ joint smoothing direction
\end{itemize}
\subsection{Self-snakes}
\label{sec-8-4}
\begin{itemize}
\item Linear diffusion does not allow edge enhancment $\rightarrow$ nonlinear diffusion was introduced
\item Same with MCM, we use same diffusivity term as in isotropic nonlinear diffusion:
\[ \partial_t u = |\nabla u| \text{div}\left( g(|\nabla u_\sigma|^2)
        \frac{\nabla u}{|\nabla u|}\right) \]
where $g$ is called \textbf{edge-stopping function}
\item introduces a \textbf{shock term}, which is responsible for edge-enhancment
(shock term means $sign(\Delta u)$ in some form)
\end{itemize}
\subsection{Geodesic active contours}
\label{sec-8-5}
\begin{itemize}
\item interactive technique
\item start with initial curve $c_0$
\item create (binary?) image $f$ with filled curve
\item evolve f with self snakes, where edge-stopping term depends on original image h
     \[ \partial_t u = |\nabla u| \left( g(|\nabla h_\sigma|^2) 
        \frac{\nabla u}{|\nabla u|} \right) \]
\item steady-state of evolution are the Euler-Lagrange equations for the energy function
\[ E_h(c):= \oint_{c(\rho)} g(|\nabla u|^2) |c_\rho(\rho)| d\rho \]
where $g$ is a weighting and $|c_\rho(\rho)|$ is arc length
\item minimizing $E_h$ means:
\begin{itemize}
\item finding curve with minimal length
\item wrt. to some image induced metric (weight)
\end{itemize}
\item curve is attracted to by image edges were weight $g$ is small
\item convergence to \textbf{next local minimum}
\end{itemize}
\subsection{Region based active contours}
\label{sec-8-6}
\begin{itemize}
\item use \textbf{region based information} instead of edge-based
\begin{itemize}
\item more robust under noise
\item convergence behaviour more global
\item include other features within region, e.g. texture features
\end{itemize}
\end{itemize}

\subsubsection{Chan-Vese Model}
\label{sec-8-6-1}
\begin{itemize}
\item Energy functional:
\[ E_{CV}(c,u_{in}, u_{out}) = \int_{x\in c} (f(x)-u_{in})^2 dx +
         \int_{x \not \in c} (f(x)-u_{out})^2 dx + \lambda l(c) \]
where $u_{in}$ is arithmetic mean of $f$ in $c$ (same with $_{out}$),
$l(c)$ is contour length
\item minimizing energy functional drives initinal curve towards segment boundaries
\item equivalent to \emph{Mumford-Shah cartoon} model with only \textbf{two} regions
\item create binary image $v$ according do inital contour $c_0$
\item new energy functional:
\[ E_I(v) = \int_\Omega (f-u_{in})^2 H(v) + (f-u_{out})^2 (1-H(v)) +
         \lambda |\nabla H(v)| dx \]
with Heavyside function $H(v)$
\end{itemize}

\subsubsection{Extension of Chan-Vese model}
\label{sec-8-6-2}
\begin{itemize}
\item so far regions modeled by their mean grey value, but also possible:
\begin{itemize}
\item higher order statistics (staticstical moments)
\item color information
\item texture models
\item motion information
\item etc.
\end{itemize}
\item one can also use probabilities to describe how likely a value of $f$ is inside
this curve
\item if shape of region is now, one can add additional energy term which describes
difference between actual shape and desired (least square)
\end{itemize}

\section{Pattern recognition}
\label{sec-9}
\subsection{Basics}
\label{sec-9-1}
\begin{itemize}
\item performance of classifier can be evaluated with
\begin{itemize}
\item standard procedure (use trainend classifier on unseen test data)
\item shuffle procedure  (shuffle test and training data)
\item k-fold cross validation
\end{itemize}
\end{itemize}
\subsection{Bayes decision theory}
\label{sec-9-2}
\subsubsection{Normal distribution}
\label{sec-9-2-1}
\begin{itemize}
\item Mahalanobis distance:
      \[ r:= \sqrt{(x-\mu)^T\Sigma^{-1}(x-\mu)} \]
\end{itemize}

\subsubsection{Bayes decision theory}
\label{sec-9-2-2}
\begin{itemize}
\item statistically \textbf{best} classifier
\item assuming priors and likelihood are known
\item choose class with highest posterior
\item if feature is not scalar, but \textbf{vector} special considerations:
\begin{itemize}
\item likelihood $P(x|w_j)=P(x_1,...,x_d|w_j)$
\item if components of $x$ are independant: $P(x_1,...,x_d|w_j)=\prod_i P(x_i|w_j)$
and effort grow \textbf{lineary}
\item if not independant $\rightarrow$ \textbf{exponentially} more trainings data is needed
\item \textbf{curse of dimensionality}
\item consider PCA
\end{itemize}
\end{itemize}

\subsubsection{Bayes risk and Bayes decision rule}
\label{sec-9-2-3}
\begin{itemize}
\item consider costs for correct and wrong answers $\rightarrow$ \textbf{loss function}
\item for each possible action associate a loss
\item yields loss function for action $a_i$ under (true) state $w_j$:
      \[ \lambda(a_i|w_j) \]
\item conditional risk:
\[ R(a_i|x) = \sum_{j=1}^c \lambda(a_i|w_j)P(w_j|x) \]
Note: we expect that costs for incorrect classifciation is higher than normal one
\item \textbf{Bayes Risk:}
\[ R= \int R(a_{k(x)}|x) \cdot p(x) dx \]
where $a_{k(x)}$ is best possible action for $x$
\item Bayes Risk ist the minimum possible risk by any classifier
\item Bayes criterion with \emph{likelihood ratio test (LRT)} for binary class case:
\[ \Lambda(x) = \frac{P(x|w_1)}{P(x|w_2} \gtrless 
         \frac{\lambda_{1,2} - \lambda_{2,2}}{\lambda_{2,1}-\lambda_{1,1}} \cdot
         \frac{P(w_2)}{P(w_1)} \]
choose $w_1$ if left side is bigger, otherwise $w_2$
\item for \emph{multi-class case} no simple LRT, normally riscs have to be computed
\begin{itemize}
\item exception: \textbf{zero-one} loss (0 cost for correct, 1 cost for false)
\item Risk: $R(a_i|x) = 1-P(w_i|x)$
\item \textbf{maximum a posterior criterion:}
        \[ \Lambda_{MAP}(x) = \frac{P(w_i|x)}{P(w_j|x)} \gtrless 1 \Leftrightarrow 
	   \frac{P(x|w_i)}{P(x|w_j)} \gtrless \frac{P(w_j)}{P(w_i)} \]
\item is called \textbf{maximum likelihood criterion} if priors are equal
\end{itemize}
\end{itemize}

\subsubsection{Discriminant based classification}
\label{sec-9-2-4}
\begin{itemize}
\item use \textbf{discriminant functions} $g$ to decides between several classes
\item choose class $i$ if $g_i(x)>g_j(x) \; \forall j, j\neq i$
\item Bayes rules: $g_i(x) = P(w_i|x)$
\item discriminant functions so far (bayes decision with risk):
\begin{itemize}
\item for Bayes classifier (with risks): $g_i(x) = -R(a_i|x)$
\item for MAP classifier (symmetrical costs): $g_i(x) = P(w_i|x)$
\item for maximum likelihood classifier (equal priors): $g_i(x) = P(x|w_i)$
\end{itemize}
\item decision boundarys of discriminant functions don't change under $f(g_i(x))$
if $f$ is monotonically increasing (e.g. \emph{logarithm})
\item considering \textbf{normally distributed} likelihoods
      \[ P(x|w_i)=\frac{1}{(2\pi)^{d/2}|\Sigma_i|^{1/2}}e^{-\frac{1}{2}(x-\mu_i)^T
         \Sigma_i^{-1}(x-\mu_i)} \]
\item resulting discriminant function is (logarithm of posterior):
      \[ g_i(x) = -\frac{1}{2}(x-\mu_i)^T \Sigma_i^{-1}(x-\mu_i) - 
         \frac{d}{2}ln(2\pi) - \frac{1}{2}ln(|\Sigma_i|)+ln(P(w_i)) \]
\end{itemize}
Three cases:
\begin{itemize}
\item $\Sigma_i = \sigma I$: 
\[ g_i(x) = -\frac{\|x-\mu_i\|^2}{2\sigma}+ln(P(w_i)) = w_i^Tx+w_{i0} \]
with $w_i = \frac{1}{\sigma}\mu_i$
\begin{itemize}
\item Minimum Distance Classifier (Euclidean)
\item linear discriminant
\end{itemize}
\item $\Sigma_i = \Sigma$ (same for all clases but arbitrary):
\[ g_i(x) = -\frac{1}{2}(x-\mu_i)^T \Sigma^{-1}(x-\mu_i) + ln(P(w_i)) = w_i^T x+w_{i0} \]
with $w_i = \Sigma^{-1}\mu_i$
\begin{itemize}
\item Minimum Distance Classifier (Mahalanobis)
\item also linear discriminant
\end{itemize}
\item $\Sigma_i$ =  Arbitrary:
\[ g_i(x) = x^T W_i x +w_i^t x + w_{i,0} \]
with $W_i = -\frac{1}{2}\Sigma_i^{-1}$, $w_i = \Sigma_i^{-1}\mu_i$
\begin{itemize}
\item disciminant is \textbf{quadratic}
\end{itemize}
\end{itemize}
\subsection{Parameter estimation}
\label{sec-9-3}
\subsubsection{Maximum likelihood estimation (MLE)}
\label{sec-9-3-1}
\begin{itemize}
\item find parameter set $\theta$, which maximize likelihood of observed data $P(D|\theta)$
\item if parameters $\theta_j$ are independant: $P(D|\theta) = \prod_{k=1}^n P(x_k|\theta)$
\item likelihood und \textbf{log-likelihood} have same maxima
      \[ ln(P(D|\theta))=ln(\prod_k p(x_k|\theta)) = \sum_k ln(P(x_k|\theta)) \]
\item find maxima by setting gradient to zero:
      \[ \sum_k \nabla_\theta ln(P(x_k|\theta))=0 \]
\item two main errors:
\begin{itemize}
\item \textbf{Bias:} How close is the estimate to the true value (\textbf{goodness})
\item \textbf{Variance:} How much would the estimate change with different dataset (\textbf{reliability})
\end{itemize}
\item often practical for most application
\end{itemize}
\subsubsection{Bayesian estimation (BE)}
\label{sec-9-3-2}
\begin{itemize}
\item Parameters $\theta$ are Random variables which have priors
\item calculate posterior $P(\theta|D)$ $\rightarrow$ gives the probability for the parameters
\end{itemize}
\subsubsection{Problems}
\label{sec-9-3-3}
\begin{itemize}
\item we assume probability distribution for parameters or model
\item often invidiual features are not independant
\item Idea: estimate \textbf{distribution function}
\item \emph{geometric approach:} take middle $x^*$ of region $R$ and estimate $p(x)= x^* vol(R)$
\item \emph{frequency approach:} number of samples in region $R$
\item Region has to be:
\begin{itemize}
\item large enough to contain sufficent amount of samples
\item small enough to justify constancy assumptions of $p(x)$ in $R$
\end{itemize}
\item LAST PARTS MISSING (neccessary conditions for R or vol or p)
\end{itemize}
\subsection{Non-parametric techniques}
\label{sec-9-4}
\subsubsection{Parzen windows}
\label{sec-9-4-1}
% Emacs 23.4.1 (Org mode 8.0.6)
\end{document}
